%%%%%%%%%%%%%%%%%
% This is an sample CV template created using altacv.cls
% (v1.3, 10 May 2020) written by LianTze Lim (liantze@gmail.com). Now compiles with pdfLaTeX, XeLaTeX and LuaLaTeX.
%
%% It may be distributed and/or modified under the
%% conditions of the LaTeX Project Public License, either version 1.3
%% of this license or (at your option) any later version.
%% The latest version of this license is in
%%    http://www.latex-project.org/lppl.txt
%% and version 1.3 or later is part of all distributions of LaTeX
%% version 2003/12/01 or later.
%%%%%%%%%%%%%%%%

%% If you need to pass whatever options to xcolor
\PassOptionsToPackage{dvipsnames}{xcolor}

%% If you are using \orcid or academicons
%% icons, make sure you have the academicons
%% option here, and compile with XeLaTeX
%% or LuaLaTeX.
% \documentclass[10pt,a4paper,academicons]{altacv}

%% Use the "normalphoto" option if you want a normal photo instead of cropped to a circle
% \documentclass[10pt,a4paper,normalphoto]{altacv}

\documentclass[10pt,a4paper,ragged2e,withhyper]{altacv}

%% AltaCV uses the fontawesome5 and academicons fonts
%% and packages.
%% See http://texdoc.net/pkg/fontawesome5 and http://texdoc.net/pkg/academicons for full list of symbols. You MUST compile with XeLaTeX or LuaLaTeX if you want to use academicons.

% Change the page layout if you need to
\geometry{left=1.25cm,right=1.25cm,top=1.5cm,bottom=1.5cm,columnsep=1.2cm}

% The paracol package lets you typeset columns of text in parallel
\usepackage{paracol}

% Change the font if you want to, depending on whether
% you're using pdflatex or xelatex/lualatex
\ifxetexorluatex
  % If using xelatex or lualatex:
  \setmainfont{Roboto Slab}
  \setsansfont{Lato}
  \renewcommand{\familydefault}{\sfdefault}
\else
  % If using pdflatex:
  \usepackage[utf8]{inputenc}
  \usepackage[T1]{fontenc}
	\usepackage[rm]{roboto}
  \usepackage[defaultsans]{lato}
  % \usepackage{sourcesanspro}
  \renewcommand{\familydefault}{\sfdefault}
\fi

% Change the colours if you want to
\definecolor{SlateGrey}{HTML}{2E2E2E}
\definecolor{LightGrey}{HTML}{666666}
\definecolor{DarkPastelRed}{HTML}{450808}
\definecolor{PastelRed}{HTML}{8F0D0D}
\definecolor{GoldenEarth}{HTML}{E7D192}
\colorlet{name}{black}
\colorlet{tagline}{PastelRed}
\colorlet{heading}{DarkPastelRed}
\colorlet{headingrule}{GoldenEarth}
\colorlet{subheading}{PastelRed}
\colorlet{accent}{PastelRed}
\colorlet{emphasis}{SlateGrey}
\colorlet{body}{LightGrey}

% Change some fonts, if necessary
\renewcommand{\namefont}{\Huge\rmfamily\bfseries}
\renewcommand{\personalinfofont}{\footnotesize}
\renewcommand{\cvsectionfont}{\LARGE\rmfamily\bfseries}
\renewcommand{\cvsubsectionfont}{\large\bfseries}


% Change the bullets for itemize and rating marker
% for \cvskill if you want to
\renewcommand{\itemmarker}{{\small\textbullet}}
\renewcommand{\ratingmarker}{\faCircle}

%% sample.bib contains your publications
\addbibresource{sample.bib}

\begin{document}
\name{Qianli Wang}
%\tagline{Your Position or Tagline Here}
%% You can add multiple photos on the left or right
\photoR{2.8cm}{identity}
% \photoL{2.5cm}{Yacht_High,Suitcase_High}

\personalinfo{%
  % Not all of these are required!
  \email{wolfgang61617@gmail.com}
  \phone{+49 015237773079}
  \mailaddress{Dauerwaldweg 1, 14055, Berlin, Germany}
  %\location{Location, COUNTRY}
  %\homepage{www.homepage.com}
  %\twitter{@qiaw99}
  %\linkedin{your_id}
  \github{qiaw99}
  %% You MUST add the academicons option to \documentclass, then compile with LuaLaTeX or XeLaTeX, if you want to use \orcid or other academicons commands.
  % \orcid{0000-0000-0000-0000}
  %% You can add your own arbtrary detail with
  %% \printinfo{symbol}{detail}[optional hyperlink prefix]
  % \printinfo{\faPaw}{Hey ho!}[https://example.com/]
  %% Or you can declare your own field with
  %% \NewInfoFiled{fieldname}{symbol}[optional hyperlink prefix] and use it:
  % \NewInfoField{gitlab}{\faGitlab}[https://gitlab.com/]
  % \gitlab{your_id}
}

\makecvheader
%% Depending on your tastes, you may want to make fonts of itemize environments slightly smaller
% \AtBeginEnvironment{itemize}{\small}

%% Set the left/right column width ratio to 6:4.
\columnratio{0.6}

% Start a 2-column paracol. Both the left and right columns will automatically
% break across pages if things get too long.
\begin{paracol}{2}
%\cvsection{Experience}

%\cvevent{Job Title 1}{Company 1}{Month 20XX -- Ongoing}{Location}
%\begin{itemize}
%\item Job description 1
%\item Job description 2
%\end{itemize}

%\divider

%\cvevent{Job Title 2}{Company 2}{Month 20XX -- Ongoing}{Location}
%\begin{itemize}
%\item Job description 1
%\item Job description 2
%\end{itemize}

\cvsection{Projects}
\cvevent{Bridge}{Implemented in Java}{https://github.com/qiaw99/bridge}{}
\begin{itemize}
    \item Simulation of the game "Bridge".
\end{itemize}

\divider

\cvevent{Roguelike}{Implemented ih Rust}{}{https://github.com/qiaw99/rust\_roguelike}
\begin{itemize}
\item 
A subgenre of role-playing video games characterized by a dungeon crawl through procedurally generated levels, turn-based gameplay, tile-based graphics, and permanent death of the player character.
\end{itemize}

\divider

\cvevent{Simple Database}{Implemented in C}{https://github.com/qiaw99/Database}{}
\begin{itemize}
    \item Simple Database which supports operations like INSERT, SELECT etc.
\end{itemize}

\divider

\cvevent{Login\_System}{Implemented in Java}{https://github.com/qiaw99/Self-Learning/tree/master/Java/$Login\_System$}{}
\begin{itemize}
    \item A simple system for Login and for booking air tickets.
\end{itemize}

\divider

\medskip

\cvsection{Skills}
\cvtag{UML}
\cvtag{OCL}\\
\cvtag{Git}
\cvtag{XML}
\cvtag{JSON}\\
\cvtag{Microsoft Office}
\cvtag{Libre Office}\\
\cvtag{Object oriented programming}
\cvtag{Unit test}\\
\cvtag{LaTeX}
\cvtag{pdfLaTeX}
\cvtag{LuaLateX}
\cvtag{BibTeX}
\cvtag{MarkDown}\\




% use ONLY \newpage if you want to force a page break for
% ONLY the current column
%\newpage



%% Switch to the right column. This will now automatically move to the second
%% page if the content is too long.
\switchcolumn

\cvsection{My Life Philosophy}

\begin{quote}
``Something smart or heartfelt, preferably in one sentence.''
\end{quote}

\cvsection{Programming Languages}
\cvtag{C}
\cvtag{Rust}
\cvtag{asm}

\divider

\cvtag{Java}
\cvtag{Python}
\cvtag{Lua}

\divider

\cvtag{Haskell}

\divider

\cvtag{Javascript}
\cvtag{HTML}
\cvtag{CSS}

\divider

\cvtag{PostgreSQL}
%\cvachievement{\faTrophy}{Fantastic Achievement}{and some details about it}

%\divider

%\cvachievement{\faHeartbeat}{Another achievement}{more details about it of course}

%\divider

%\cvachievement{\faHeartbeat}{Another achievement}{more details about it of course}

\cvsection{Strengths}

\cvtag{Hard-working}
\cvtag{Eye for detail}\\
\cvtag{Highly motivated}

\cvsection{Languages}

\cvskill{English}{4}
\divider

\cvskill{Chinese}{5}
\divider

\cvskill{German}{4}

%% Yeah I didn't spend too much time making all the
%% spacing consistent... sorry. Use \smallskip, \medskip,
%% \bigskip, \vpsace etc to make ajustments.
\medskip

\cvsection{Education}
\cvevent{Studienkolleg}{Niedersächsisches Studienkollge}{Oct.2017 -- July.2018}{}
\divider
\cvevent{B.Sc.\ Computer Science}{Freie Universität Berlin}{Oct. 2018 -- Now}{}



\cvsection{Score}
1,7 out of 6,0 (In germany grade system: 1,0 is the best and 6,0 is the worst)
% \cvref{name}{email}{mailing address}



\end{paracol}


\end{document}
